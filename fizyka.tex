\documentclass{article}

    \usepackage{polski}
    \usepackage[utf8]{inputenc}

\title{Fizyka - notatki}   
\author{Maciej Gicala}

\begin{document}
    \maketitle
    \section{Termodynamika}
        \subsection{Zerowa zasada termodynamiki}
        Jeśli układy A i B mogące ze sobą wymieniać ciepło są ze sobą w równowadze 
        termicznej, i to samo jest prawdą dla układów B i C, to układy A i C również 
        są ze sobą w równowadze termicznej.
        \subsection{Pierwsza zasada termodynamiki}
            Różnica pomiędzy ciepłem dostarczonym do układu, a pracą wykonaną przez ten 
            układ jest równa zmianie energii wewnętrznej układu.
            \begin{equation}
                \Delta E_w = Q - W
            \end{equation}
            \paragraph{Szczególne przypadki:}
                \subparagraph{adiabatyczne}
                \begin{equation}
                    Q = 0
                \end{equation}
                \subparagraph{stała objętość}
                \begin{equation}
                    W = 0
                \end{equation}
                \subparagraph{cykl zamknięty}
                \begin{equation}
                    \Delta E_w = 0
                \end{equation}
                \subparagraph{rozprężenie swobodne}
                \begin{equation}
                    Q = W = 0
                \end{equation}
        \newpage
        \subsection{Przewodnictwo cieplne}
            \begin{equation}
                P = \frac{Q}{t} = k \cdotp S \cdotp \frac{\Delta T}{L}
            \end{equation}
            \textbf{Gdzie:}\\
                P - strumień ciepła \\
                Q - ciepło  \\
                t - czas\\
                k - współczynnik przewodnictwa cieplnego\\
                S - pole powierzchni styku\\
                $\Delta$T - zmiana temperatury\\
                L - długość przewodnika\\

\end{document}\maketitle
