\documentclass{article}

    \usepackage{polski}
    \usepackage[utf8]{inputenc}

\title{Fizyka - notatki}   
\author{Maciej Gicala}

\begin{document}
    \maketitle
    \section{Termodynamika}
        \subsection{Zerowa zasada termodynamiki}
        Jeśli układy A i B mogące ze sobą wymieniać ciepło są ze sobą w równowadze 
        termicznej, i to samo jest prawdą dla układów B i C, to układy A i C również 
        są ze sobą w równowadze termicznej.
        \subsection{Pierwsza zasada termodynamiki}
            Różnica pomiędzy ciepłem dostarczonym do układu, a pracą wykonaną przez ten 
            układ jest równa zmianie energii wewnętrznej układu.
            \begin{equation}
                \Delta E_w = Q - W
            \end{equation}
            \paragraph{Szczególne przypadki:}
                \begin{itemize}
                    \item adiabatyczne
                    \begin{equation}
                        Q = 0
                    \end{equation}
                    \item stała objętość
                    \begin{equation}
                        W = 0
                    \end{equation}
                    \item cykl zamknięty
                    \begin{equation}
                        \Delta E_w = 0
                    \end{equation}
                    \item rozprężenie swobodne
                    \begin{equation}
                        Q = W = 0
                    \end{equation}
                \end{itemize}
        \newpage
        \subsection{Przewodnictwo cieplne}
            \begin{equation}
                P = \frac{Q}{t} = k \cdotp S \cdotp \frac{\Delta T}{L}
            \end{equation}
            \textbf{Gdzie:}
                \begin{itemize}
                    \item P - strumień ciepła
                    \item Q - ciepło
                    \item t - czas
                    \item k - współczynnik przewodnictwa cieplnego
                    \item S - pole powierzchni styku
                    \item $\Delta$T - zmiana temperatury
                    \item L - długość przewodnika
                \end{itemize}

    \newpage
    \section{Elektrostatyka, elektrodynamika}
        \paragraph{Pole}
            to obszar w przestrzeni, w którym w każdym punkcie można
            opisać jakieś oddziaływanie.
            Skalarne(temperatura), Wektorowe(grawitacja).

        \paragraph{Przewodnik}
            to ciało umożliwiające ruch ładunków w całej objętości.

        \paragraph{Izolator}
            to ciało nieposiadające wolnych ładunków.
        

        \
        
        \textit{Wyróżniamy dodatkowo nadprzewodniki(przewodzą
        ładunki bezstratnie) i półprzewodniki(przewodzą
        tylko w jednym kierunku).}

        \subsection{Prawo Coulomba}
            Jeżeli dwa ładunki punktowe $q_1$ i $q_2$ znajdują się
            w odległości r to siła elektrostatyczna pomiędzy nimi jest
            dana wzorem:
            \begin{equation}
                F = k \cdotp \frac{q_1 \cdotp q_2}{r^2}
            \end{equation} 
            Siła przyciąga ładunki różnoimienne, a odpycha jednoimienne
        \\
        \textbf{Gdzie:}
        \begin{itemize}
            \item $k = \frac{1}{4 \pi \cdotp \varepsilon_0} = 
            8,99 \cdotp 10^9 \frac{N \cdotp m^2}{C^2}$
            - stała Coulomba
        
            \item $\varepsilon_0 = 8,85 \cdotp 10^{-12} \frac{C^2}{N \cdotp m^2}$
            - przenikalność elektryczna prózni
        \end{itemize}
        
        \paragraph{Ładunek elementarny }
        $e = 1,6 \cdotp 10^{-19} $C
        \paragraph{Kulomb} 
        $C = 6,24 \cdotp 10^{18} $e

        \paragraph{Pole elektryczne }
        to pole wektorowe występujące wokół każdego ładunku

        \
        
        Natężenie pola: 
        \begin{equation}
            \vec{E} = \frac{\vec{F}}{q_0} = k \cdotp \frac{q}{r^2}    
        \end{equation}
        

        \subsection{Prawo Gaussa}
            Prawo Gaussa określa związek między natężeniem pola
            elektrycznego w punktach na zamkniętej powierzchni 
            Gaussa i całkowitym ładunkiem objętym tą przestrzenią.
            \
            
            Strumień elektryczny $\Phi$ przenikający przez powierzchnię
            Gaussa jest proporcjonalny do całkowitej liczby linii pola
            elektrycznego przechodzących przez tę powierzchnię.

            \

            Definicja strumienia:
            $\Phi = v\cdotp S \cdotp \cos\theta = \vec{v} \cdotp 
            \vec{S}$

            \
            
            Strumień przenikający przez powierzchnię Gaussa:
            $\Phi = \sum\nolimits_{}\vec{E} \cdotp \Delta S$

            \

            Strumień natężenia pola elektrycznego przenikający przez
            dowolną powierzchnię zamkniętą w jednorodnym środowisku
            o bezwzględnej przenikalności elektrycznej $\varepsilon$
            , jest równy stosunkkowi całkowitego ładunku znajdującego
            się wewnątrz tej powierzchni do wartości tejże przenikalności
            
            \
            
            \begin{equation}
                \Phi = \oint\nolimits_{} \vec{E}\cdotp\Delta S
            \end{equation}
            \textit{definicja użyteczna obliczeniowo}
        

\end{document}\maketitle
