\documentclass{article}

    \usepackage{polski}
    \usepackage[utf8]{inputenc}

\title{Fizyka - notatki}   
%\author{Maciej Gicala}

\begin{document}
    \maketitle
    \section{Termodynamika}
        \subsection{Zerowa zasada termodynamiki}
        Jeśli układy A i B mogące ze sobą wymieniać ciepło są ze sobą w równowadze 
        termicznej, i to samo jest prawdą dla układów B i C, to układy A i C również 
        są ze sobą w równowadze termicznej.
        \subsection{Pierwsza zasada termodynamiki}
            Różnica pomiędzy ciepłem dostarczonym do układu, a pracą wykonaną przez ten 
            układ jest równa zmianie energii wewnętrznej układu.
            \begin{equation}
                \Delta E_w = Q - W
            \end{equation}
            \paragraph{Szczególne przypadki:}
                \begin{itemize}
                    \item adiabatyczne
                    \begin{equation}
                        Q = 0
                    \end{equation}
                    \item stała objętość
                    \begin{equation}
                        W = 0
                    \end{equation}
                    \item cykl zamknięty
                    \begin{equation}
                        \Delta E_w = 0
                    \end{equation}
                    \item rozprężenie swobodne
                    \begin{equation}
                        Q = W = 0
                    \end{equation}
                \end{itemize}
        \newpage
        \subsection{Przewodnictwo cieplne}
            \begin{equation}
                P = \frac{Q}{t} = k \cdotp S \cdotp \frac{\Delta T}{L}
            \end{equation}
            \textbf{Gdzie:}
                \begin{itemize}
                    \item P - strumień ciepła
                    \item Q - ciepło
                    \item t - czas
                    \item k - współczynnik przewodnictwa cieplnego
                    \item S - pole powierzchni styku
                    \item $\Delta$T - zmiana temperatury
                    \item L - długość przewodnika
                \end{itemize}

    \newpage
    \section{Elektrostatyka, elektrodynamika}
        \paragraph{Pole}
            to obszar w przestrzeni, w którym w każdym punkcie można
            opisać jakieś oddziaływanie.
            Skalarne(temperatura), Wektorowe(grawitacja).

        \paragraph{Przewodnik}
            to ciało umożliwiające ruch ładunków w całej objętości.

        \paragraph{Izolator}
            to ciało nieposiadające wolnych ładunków.
        

        \
        
        \textit{Wyróżniamy dodatkowo nadprzewodniki(przewodzą
        ładunki bezstratnie) i półprzewodniki(przewodzą
        tylko w jednym kierunku).}

        \subsection{Prawo Coulomba}
            Jeżeli dwa ładunki punktowe $q_1$ i $q_2$ znajdują się
            w odległości r to siła elektrostatyczna pomiędzy nimi jest
            dana wzorem:
            \begin{equation}
                F = k \cdotp \frac{q_1 \cdotp q_2}{r^2}
            \end{equation} 
            Siła przyciąga ładunki różnoimienne, a odpycha jednoimienne
        \\
        \textbf{Gdzie:}
        \begin{itemize}
            \item $k = \frac{1}{4 \pi \cdotp \varepsilon_0} = 
            8,99 \cdotp 10^9 \frac{N \cdotp m^2}{C^2}$
            - stała Coulomba
        
            \item $\varepsilon_0 = 8,85 \cdotp 10^{-12} \frac{C^2}{N \cdotp m^2}$
            - przenikalność elektryczna prózni
        \end{itemize}
        
        \paragraph{Ładunek elementarny }
        $e = 1,6 \cdotp 10^{-19} C$
        \paragraph{Kulomb} 
        $C = 6,24 \cdotp 10^{18} e$

        \paragraph{Pole elektryczne }
        to pole wektorowe występujące wokół każdego ładunku

        \
        
        Natężenie pola: 
        \begin{equation}
            \vec{E} = \frac{\vec{F}}{q_0} = k \cdotp \frac{q}{r^2}    
        \end{equation}
        

        \subsection{Prawo Gaussa}
            Prawo Gaussa określa związek między natężeniem pola
            elektrycznego w punktach na zamkniętej powierzchni 
            Gaussa i całkowitym ładunkiem objętym tą przestrzenią.
            \
            
            Strumień elektryczny $\Phi$ przenikający przez powierzchnię
            Gaussa jest proporcjonalny do całkowitej liczby linii pola
            elektrycznego przechodzących przez tę powierzchnię.

            \

            Definicja strumienia:
            $\Phi = v\cdotp S \cdotp \cos\theta = \vec{v} \cdotp 
            \vec{S}$

            \
            
            Strumień przenikający przez powierzchnię Gaussa:
            $\Phi = \sum\nolimits_{}\vec{E} \cdotp \Delta S$

            \

            Strumień natężenia pola elektrycznego przenikający przez
            dowolną powierzchnię zamkniętą w jednorodnym środowisku
            o bezwzględnej przenikalności elektrycznej $\varepsilon$
            , jest równy stosunkkowi całkowitego ładunku znajdującego
            się wewnątrz tej powierzchni do wartości tejże przenikalności
            
            \
            
            \begin{equation}
                \Phi = \oint\nolimits_{} \vec{E}\cdotp\Delta S
            \end{equation}
            \textit{definicja użyteczna obliczeniowo}
            
            \paragraph{Prawo Gaussa i prawo Coulomba} 
            \

            W przypadku ładunku
            punktowego zamkniętego w sferycznej powierzchni
            Gaussa wektor pola jest prostopadły do powierzchni.

            \begin{equation}
                \varepsilon_0 \cdotp \oint\nolimits_{} \vec{E} d\vec{S}
                = \varepsilon_0 \cdotp \oint\nolimits_{} E \cdotp dS
                = q_{wewnetrzne}    
            \end{equation}

            Po obliczeniu całki po powierzchni sferycznej:
            \begin{equation}
                \varepsilon_0 \cdotp E \cdotp 4\pi \cdotp r^2 = q
                \Longrightarrow
                E = \frac{q}{\varepsilon_0 \cdotp 4 \pi r^2}
            \end{equation}

            \paragraph{Potencjałem elektrycznym}
            w dowolnym punkcie pola elektrycznego E nazywa się
            stosunek pracy W wykonanej przez siłę elektryczną przy
            przenoszeniu ładunku q z tego punktu do nieskończoności,
            do wartości tego ładunku.

            \begin{equation}
                V = -\frac{W}{q}
            \end{equation}

            Pracę oblicza się całkując pole elektryczne po przesunięciu
            \begin{equation}
                W = q_0 \cdotp \oint\limits_{pocz}^{konc} {\vec{E}
                d\vec{S}}
                \Longrightarrow
                V = -\oint\nolimits_{} \vec E d\vec S
            \end{equation}

            Przy założeniu pola jednorodnego:
            \begin{equation}
                E = - \frac{\Delta V}{\Delta s}
            \end{equation}
            
            Bezwzględna wartość różnicy potencjałów oznaczana jest
            jako U.

            \
            
            Różnica potencjałów na okładkach kondensatora jest 
            proporcjonalna do ładunku i dana jest wzorem:
            \begin{equation}
                q = C \cdotp U
            \end{equation}
            \
            
            C - pojemność kondensatora
            
            \
            
            Jednostką pojemności jest \textbf{Farad}

            \
            
            $[F] = [\frac{C}{V}]$

            Pojemność kondensatora płaskiego:
            \begin{equation}
                C = \frac{\varepsilon_0 \cdotp S}{d}
            \end{equation}

            \paragraph{Łączenie kondensatorów:}
            \begin{itemize}
                \item równolegle - na każdym kondensatorze występuje
                taka sama różnica potencjałów.
                \begin{equation}
                    C_{calk} = \frac{q_c}{U} = \sum\limits_{i = 1}^n C_i
                \end{equation}

                \item szeregowo - na każdym z kondensatorów odkłada
                się część różnicy potencjałów.
                \begin{equation}
                    \frac{1}{C_{calk}} =  \sum\limits_{i = 1}^n \frac{1}{C_i}
                \end{equation}
            \end{itemize}

        \subsection{Prąd elektryczny}
            \textbf{Prąd elektryczny} to uporządkowany ruch ładunków elektrycznych
            \paragraph{Natężenie} to wielkość charakteryzująca przepływ prądu
            zdefiniowana jako stosunek wartości ładunku elektrycznego
            przepływającego przez wyznaczoną powierzchnię do czasu
            przepływu ładunku.
            \begin{equation}
                I = \frac{dq}{dt}
            \end{equation}

            \

            Natężenie prądu elektrycznego oznaczamy literą I w przypadku
            prądu stałego, a literą i gdy dotyczy prądu zmiennego.

            \

            Umowny kierunek prądu: od bieguna dodatniego do ujemnego.

            \

            Prąd w pojedynczym obwodzie ma jedno natężenie na całej długości.

            \
            
            \textbf{Rzeczywiście poruszające się ładunki elektryczne}
            mogą posiadać kierunek ruchu niezgodny z umownym kierunkiem
            prądu.

            \paragraph{Przypadki: }
            \begin{itemize}
                \item W metalach nośnikami prądu są elektrony, a więc
                kierunek ich dryfu jest dokładnie przeciwny do umownego
                kierunku prądu
                \item W elektrolitach nośnikami prądu są jony -
                zarówno dodatnie, jak i ujemne. Jony poruszają się w
                przeciwnych kierunkach, jednak prądy jakie są z nimi
                związane dodają się, bo prąd jonów ujemnych jest
                traktowany jako przeciwny do ich ruchu.
                \item W połprzewodnikach nośnikami mogą być zarówno
                elektrony ujemne jak i dodatnie dziury(braki elektronów).
                Występuje sytuacja podobna do ruchu jonów - dziury tworzą
                prąd zgodny z ich kierunkiem ruchu, prąd elektronowy jest
                przeciwny do ruchu ładunków go tworzących.
                \item Ładunki poruszające się w próżni też tworzą prąd elektryczny
                na zasadzie strumienia np. wiązka elektronów w lampie kineskopowej
                biegnie od działa elektronowego do ekranu. Tutaj też, z racji,
                że elektrony są ujemne, prąd tej wiązki płynie od ekranu do
                działa elektronowego.
            \end{itemize}

            \paragraph{I. Prawo Kirchoffa}
            
            \
            
            Suma natężeń prądów wpływających do węzła jest równa sumie
            natężeń prądów wypływających z tego węzła.
            
            \
            
            \textit{Pierwsze prawo Kirchoffa jest nazywane również prawem węzła}
            \paragraph{Opór elektryczny przewodnika} jest to spadek
            energii nośników prądu na skutek kolizji zachodzących 
            w strukturze przewodnika. Tworzona energia zamienia się
            w ciepło, ale nie następuje zmiana ilości nośników.

            \paragraph{Prawo Ohma}

            \
            
            Stosunek napięcia na końcach przewodnika do natężenia
            prądu przezeń płynącego jest określany mianem oporu
            elektrycznego.

            \begin{equation}
                R = \frac{U}{I}
            \end{equation}

            Jednostką oporu elektrycznego jest Om - $[\Omega]$

            \paragraph{Przewodnictwo elektryczne}

            \
            
            Stosunek prądu płynącego przez przewodnik do napięcia
            pomiędzy jego końcami jest stały. Wartość tego stosunku
            to przewodnictwo elektryczne.

            \begin{equation}
                G = \frac{I}{U}
            \end{equation}

            Jednostką przewodnictwa elektrycznego jest Simens - [S]

            \paragraph{Łączenie oporników:}
            \begin{itemize}
                \item szeregowo
                \begin{equation}
                    R_{z} = \sum\limits_{i = 1}^n R_i
                \end{equation}

                \item szeregowo
                \begin{equation}
                    \frac{1}{R_{z}} =  \sum\limits_{i = 1}^n \frac{1}{R_i}
                \end{equation}
            \end{itemize}
            
            \paragraph{II. Prawo Kirchoffa}
            
            \
            
            W obwodzie zamkniętym suma spadków napięć na wszystkich
            odbiornikach prądu musi być równa sumie napięć na źródłach
            napięcia.

            \
            

            \textbf{Praca prądu elektrycznego:}

            \begin{equation}
                W = U \cdotp I \cdotp t
            \end{equation}

            \textbf{Moc prądu elektrycznego:}
            
            \begin{equation}
                P = U \cdotp I = I^2 \cdotp R = \frac{U^2}{R}
            \end{equation}

            \paragraph{Półprzewodnik}
            jest to materiał wykazujący w określonych warunkach
            własności izolatora, a w innych właściwości przewodnika prądu.
            Pod względem cemicznym półprzewodniki w zasadzie nie posiadają
            wolnych elektronów ostatniej powłoki(dlatego są izolatorami),
            ale w warunkach odmiennych, przy dostarczeniu ich atomom
            dużej energii przez podgrzanie lub oświetlenie, pojawiają się
            wolne elektrony, opór ciała spada i w rezultacie prąd elektryczny
            może płynąć.

            \paragraph{Dziura elektronowa}
            to brak elektronu w pełnym paśmie walencyjnym.
            Pochodzi stąd, iż gdy w paśmie walencyjnym brakuje pojedynczego
            elektronu, występująca ,,dziura" zachowuje się jako dodatni
            nośnik ładunku elektrycznego. 

            \paragraph{Wiązanie kowalencyjne}
            to rodzaj wiązania chemicznego. Istotą tego wiązania jest
            istnienie par elektronów, które są wpółdzielone w porównywalnym
            stopniu przez oba atomy tworzące to wiązanie.

            \paragraph{Półprzewodnik samoistny}
            to półprzewodnik, którego materiał jest idealnie czysty.
            Koncentracja wolnych elektronów w półprzewodnikach jest równa
            koncentracji dziur. Półprzewodniki samoistne mają mało par
            dziura-elektron i dlatego ich opór właściwy jest duży.

            \paragraph{Półprzewodnik domieszkowany}
            Domieszkowanie polega na wprowadzeniu i aktywowaniu atomów
            domieszek do struktury kryształu. Domieszki są atomami pierwiastków
            niewchodzących w skład półprzewodnika samoistnego, ponieważ
            w wiązaniach kowalencyjnych bierze udział ustalona liczba elektronów,
            zamiana któregoś z atomów struktury na odpowiedni atom domieszki
            powoduje wystąpienie nadmiaru lub niedoboru elektronów.


            \
            
            \textbf{Domieszkowanie: }
            \begin{itemize}
                \item donorowe - Wprowadzenie domieszki dającej nadmiar
                elektronów(w stosunku do półprzewodnika samoistnego)
                powoduje powstanie półprzewodnika typu n, domieszka taka
                zaś nazywana jest donorową(oddaje elektron).
                W takim półprzewodniku powstaje dodatkowy poziom energetyczny
                (poziom donorowy) położony w obszarze pasma wzbronionego niewiele
                poniżej poziomu przewodnictwa, lub na takim samym paśmie przewodnictwa.
                Nadmiar elektronów jest uwalniany do pasma przewodnictwa(prawie pustego
                w przypadku półprzewodnika samoistnego) w postaci elektronów swobodnych
                zdolnych do przewodzenia prądu. Mówimy wtedy o przewodnictwie
                elektronowym lub przewodnictwie typu n.

                \item akceptorowe - Wprowadzenie domieszki dającej niedobór
                elektronów(w stosunku do półprzewodnika samoistnego) powoduje
                powstanie półprzewodnika typu p, domieszka taka zaś nazywa się
                domieszką akceptorową(przyjmuje elektron). W takim półprzewodniku
                powstaje dodatkowy poziom energetyczny(poziom akceptorowy).
                Poziomy takie wiążą elektrony znajdujące się na pasmie walencyjnym
                (prawie zapełnionym w przypadku półprzewodników samoistnych),
                powodując powstanie w nim wolnych miejsc(dziur elektronowych).
                Zachowują się one jak swobodne cząstki o ładunktu dodatnim i są
                zdolne do przewodzenia prądu. Mówimy o przewodnictwie dziurowym,
                lub przewodnictwie typu p.
            \end{itemize}

            Dziury ze względu na swoją masę efektywną, zwykle większą
            od masy efektywnej elektronów, mają mniejszą ruchliwość,
            przez co rezystywność materiałów typu p jest z reguły większa
            niż materiałów typu n mających ten sam poziom domieszkowania.

            \paragraph{Masa efektywna}
            to odpowienik masy dla ciał(cząstek) znajdujących się w środowisku
            materialnym, z którym oddziałują. Stosując masę efektywną w równaniach
            ruchu, automatycznie uwzględnia się obecność otaczających pól
            bez potrzeby ich dokładnej analizy. Masa efektywna może być zarówno
            mniejsza, jak i większa od masy spoczynkowej tego samego ciała
            w próżni. Może być ujemna.

            \paragraph{Diody}

            \
            
            Istotą działania większości diod jest przewodzenie prądu
            w jednym kierunku i znaczne blokowanie jego przepływu
            w drugim. Właściwości te wykorzystujemy do prostowania
            napięcia przemiennego oraz demodulacji sygnałów w
            odbiornikach radiowych. Wadą diod jest tzw. prąd wsteczny,
            prąd upływu.

            \paragraph{Tranzystor bipolarny}
            półprzewodnikowy element elektroniczny, mający zdolność wzmacniania sygnału.
            Zbudowany jest z trzech warstw półprzewodnika o różnym typie przewodnictwa.
            Charakteryzuje się tym, że niewielki prąd płynący pomiędzy dwoma jesgo elektrodami
            (nazywanymi \textit{bazą} i \textit{emiterem}) steruje większym prądem
            płynącym między emiterem, a trzecią elektrodą(nazywaną \textit{kolektorem}).
            Tranzystor bipolarny składa się z trzech warstw półprzewodnika
            o różnym typie przewodnictwa(\textbf{p-n-p} lub \textbf{n-p-n}).

            \
            
            Poszczególne warstwy noszą nazwy:
            \begin{itemize}
                \item emiter(E) - warstwa silnie domieszkowana
                \item baza(B) - warstwa cienka i słabo domieszkowana
                \item kolektor(C)
            \end{itemize}

            \

            Tranzystor bipolarny składa się z dwóch złączy p-n:
            \begin{itemize}
                \item baza-emiter(złącze emitera)
                \item baza-kolektor(złącze kolektora)
            \end{itemize}

            \paragraph{Tranzystor polowy}
            Sterowanie prądu tranzystora polowegoodbywa się za pomocą pola
            elektrycznego. Zasadniczą częścią tranzystora polowego jest krzyształ
            odpowiednio domieszkowanego półprzewodnika z dwiema elektrodami:
            \begin{itemize}
                \item źródłem(S - odpowiednik emitera)
                \item drenem(D - odpowiednik kolektora)
            \end{itemize}

            \
            
            Pomiędzy źródłem i drenem tworzy się tzw. kanał, którym płynie prąd.
            Wzdłuż kanału zmieniona jest trzecia elektroda, zwana bramką(G - odpowiednik bazy).
            Przyłożone do bramki napięcie wywołuje w krysztale dodatkowe pole elektryczne,
            które wpływa na rozkład nośników prądu w kanale. Skutkiem tego jest zmiana
            efektywnego przekroju kanału, co objawia się jako zmiana oporu pomiędzy drenem i źródłem.


            \paragraph{Siła Lorentza} to siła jaka działa na cząstkę obdarzoną ładunkiem
            elektrycznym, poruszającą się w polu elektromagnetycznym.
            
            \begin{equation}
                \vec{F} = q(\vec{E} + \vec{v} \times \vec{B})     
            \end{equation}

            \begin{itemize}
                \item $\vec{E}$ - wektor natężenia
                \item $\vec{B}$ - wektor pola magnetycznego
            \end{itemize}

            \paragraph{Prawo Amp\'ere'a}
            Całka krzywoliniowa wektora indukcji magnetycznej, wytworzonego przez stały
            prąd elektryczny w przewodniku wzdłuż linii zamkniętej otaczającej prąd, jest 
            równa sumie algebraicznej natężeń prądów przepływających(strumień gęstości
            prądu) przez dowolną powierzchnię objętą przez tę linię.

            \begin{equation}
                \oint\nolimits_{} \vec{B} d\vec{l} = \mu_0\cdotp I
            \end{equation}
            
            \
            
            \textit{Dla próżni}

            \begin{equation}
                \oint\limits_{C} \vec{H} d\vec{l} = \int\nolimits_{} \vec{J}d\vec{a} = I
            \end{equation}
                        
            \
            
            \textit{Dla dowolnego ośrodka}

            \
            
            Gdzie:
            \begin{itemize}
                \item C - linia zamknięta
                \item $\vec{H}$ - natężenie pola magnetycznego
                \item $\vec{J}$ - gęstość prądu przepływającego przez element
                $d\vec{a}$ powierzchni S, zamkniętej przez krzywą C
            \end{itemize}

            Równoważną formą prawa w postaci różniczkowej jest:
            \begin{equation}
                \nabla \times \vec{H} = \vec{J}
            \end{equation}

            Natężenie pola magnetycznego H może być wyrażone jako indukcja magnetyczna B:
            \begin{equation}
                \vec{B} = \mu \cdotp \vec{H}
            \end{equation}

            Indukcja magnetyczna dana jest wzorem:
            \begin{equation}
                \oint\limits_{C} Hdl = I
            \end{equation}

            Gdzie:
            \begin{itemize}
                \item H - natężenie pola magnetycznego
                \item I - prąd przepływający przez dowolną powierzchnię
                rozpiętą na zamkniętym konturze C
            \end{itemize}

            \paragraph{Prawo Faradaya}
            W zamkniętym obwodzie znajdującym się w \textbf{zmiennym}
            polu magnetycznym pojawia się siła elektromotoryczna
            indukcji równa szybkości zmian strumienia indukcji pola magnetycznego
            przechodzącego przez powierzchnię rozpiętą na tym obwodzie.

            \begin{equation}
                \varepsilon = -\frac{d\Phi_B}{dt}
            \end{equation}

            Gdzie:
            \begin{itemize}
                \item $\Phi_B$ - strumień indukcji magnetycznej
            \end{itemize}

            Jezeli w miejscu pętli umieści się zakmnięty przewodnik o oporze R,
            wówczas w obwodzie tego przewodnika popłynie prąd o natężeniu:
            \begin{equation}
                I = \frac{U}{R} \Longrightarrow I = \frac{\varepsilon}{R} \Longrightarrow
                I = \frac{\frac{d\Phi_B}{dt}}{R}
            \end{equation}

            \paragraph{Reguła Lenza}
            Prąd indukcyjny(nazywany też prądem wtórnym) wzbudzony w przewodniu pod wpływem
            zmiennego pola magnetycznego ma zawsze taki kierunek, że wytworzone wtórne
            pole magnetyczne przeciwdziała przyczynie(czyli zmianie pierwotnego pola magnetycznego),
            która go wywoła.

            \
            
            Wnioski:
            \begin{itemize}
                \item Jeżeli zamknięta zwojnica porusza się względem magnesu to wokół zwojnicy
                powstaje takie pole magnetyczne, które przeciwdziała temu ruchowi
                \item Jeżeli natężenie pola magnetycznego pośrodku zwojnicy wzrasta, indukuje
                to w niej pole i prąd elektryczny o takim kierunku, że wytwarzane przezeń wtórne
                pole magnetyczne przeciwdziałać będzie dalszemu wzrostowi pierwotnemu pola.
                \item Jeżeli natężenie pola magnetycznego pośrodku zwojnicy słabnie, indukuje to w
                niej pole i prąd elektryczny o takim kierunku, że wytwarzane przezeń wtórne
                pole magnetyczne podtrzymać będzie słabnące pole pierwotne
                \item Jeżeli cząstka obdarzona ładunkiem elektrycznym porusza się w polu magnetycznym
                o wzrastającym natężeniu, to ruch tej cząstki wywołuje wzrost natężenia pola magnetycznego
                przed cząstką, a osłabienie za cząstką(przeciwdziałanie zmianie pola w miejscu, gdzie jest cząstka)
                , a przy polu w ruchu słabnącym odwrotnie. Np. deformacja pola magnetycznego
                Ziemii przez wiatr słoneczny.
                \item Jeżeli cząstka obdarzona ładunkiem porusza się wzdłuż zakrzywionej linii pola
                magnetycznego, to indukowane pole zmniejsza krzywiznę tej linii 
            \end{itemize}



\end{document}\maketitle
